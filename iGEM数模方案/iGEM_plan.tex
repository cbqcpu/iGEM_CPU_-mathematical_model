% !Mode:: "TeX:UTF-8"  
%!TEX program = xelatex  
\documentclass[UTF8]{ctexart}  
\usepackage[]{geometry}
\usepackage[]{xeCJK}%---------------------------使用xeCJK宏包  
%\textwidth=420pt
\geometry{a4paper, scale=0.76, centering}
\begin{document}
	\title{2017 CPU iGEM 建模方案(初稿)}
	\author{iGEM CPU 数模组}
	\maketitle
\section{问题重述}
如图所示,系统由两部分组成,一部分为构造的Syn-notch系统,另一部分为CAR系统。系统包含的生物学过程有以下几种:
\begin{enumerate}
	\item[(1)] 受体配体相互作用(IL-17A与受体的结合)
	\item[(2)] 受体配体结合后的信号转导动力学(VP64的释放)
	\item[(3)] 基因表达建模
\begin{itemize}
\item	启动子识别
\item	转录的动力学过程
\item	翻译的动力学过程
\end{itemize}
    \item[(4)] 酶动力学过程:
\begin{itemize}
	\item 蛋白的泛素化-去泛素化过程
\end{itemize}
\end{enumerate}

针对以上所述的生物学过程主要选取基因表达建模与酶动力学过程进行定量分析试图通过实验数据确立相关模型的参数。
问题:
\begin{enumerate}
\item[(1)] 通过建立数学模型对生物学系统的一些性质进行预测?例如稳定性,鲁棒性,安全性等。
\item[(2)] 如何通过数学模型对生物学系统进行优化。
\end{enumerate}
\section{问题分析}
针对上述问题需要根据不同的过程逐一进行分析和建模。我们试图寻找一个相对合理但又不失创新性的方案。

在项目的开始,我们首先问自己:
\begin{enumerate}
\item[(1)] 这样的数学模型要解决什么样的问题?
\item[(2)] 这样的数学模型对于这个问题有什么意义?
\item[(3)] 这样的数学模型是否足够解决上述问题?
\item[(4)] 怎样去建立这样的数学模型?
\end{enumerate}

设计合成生物系统的目的是为了获得目的产物或崭新的生物功能,因此生物模块作为一个功能单元难免被植入到异源或同源生物体中,且必然会受到生物体中诸多信号的影响;合成生物学强调生物模块的“即插即用”性,要保证设计好的生物模块被方便的整合入其他功能模块中,生物模块就必须具有很好的稳定性和抗各种外界干扰的鲁棒性。此外,由于各种蛋白质、激酶自身的降解,为了维持宿主细胞内环境中各种蛋白质、代谢产物的平衡,生物模块必须具有很好的快速响应特性,能够以足够快的速度接收输入信号做出反应。
因此,分析生物模块的稳定性、鲁棒性和快速响应特性等属性,总结和抽提设计的普遍原则,对于生物模块的功能扩展和生物系统的工程化至关重要。

针对上述生物学过程逐一分析:
\subsection{受体与配体的结合及其信号转导动力学}
\begin{enumerate}
\item[(1)] 受体配体相互作用(IL-17A与受体的结合)
\item[(2)] 受体配体结合后的信号转导动力学(VP64的释放)
\end{enumerate}
\begin{tabbing}
	图片
\end{tabbing}

针对syn-notch蛋白在受到信号刺激后与其相连的Gal-VP64 脱落这一过程可以抽象成一组微分方程(ODE)。
$$\frac{dR}{dt} = f(S,r)$$   
这里S为信号强度(例如IL-17A 浓度,R等于输出信号强度(VP64浓度),r为受体浓度)

对于这样一个ODE系统建模时我们最关心的问题就是参数的选取问题,关于这一问题可以尝试先从文献或数据库中查询相关数据再参考2016年曼彻斯特大学数学建模部分所使用的方法:
基于参数检验和非参数检验的参数拟合方法。


\subsection{基因表达建模}
基因表达建模是科学领域中通过不同的技术获得结果的一个例子。基因表达的结果和动力学通过布尔网络、贝叶斯网络、有向图、常微分方程与偏微分方程系统、随机方程和基于规则的形式方法等手段进行了数学上的描述。
\subsubsection{启动子识别}
	针对我们设计的启动子查询相关的生物信息学数据库,主要包括启动子强度以及转录因子与启动子序列结合的相关参数。
\subsubsection{转录和翻译的动力学过程}
	转录和翻译的动力学过程类似之前提到的受体配体相互作用模型,同样适用ODE系统进行描述,只是方程形式稍有区别。查阅相关书籍和文献后可知真核基因表达特定过程的建模可以使用多种方法,但针对定量模型ODE系统具有不可替代的优势,因而较多的被使用。
	ODE系统中的参数确定问题(主要以米氏动力学为主)同样也是较为困难的,通常以查阅文献和数据库的方式,但经过分析后发现使用数据挖掘方法辅助确定参数比较合理与准确(参见具体建模过程)。
	

\subsection{蛋白的泛素化-去泛素化过程}

usp7
ub-foxp3 <==> foxp3
stub1


此部分为我们建模的重点,主要建立不同机理下的酶促反应动力学模型。可尝试使用数据对机理进行探究或通过实验数据拟合动力学参数。


\end{document}